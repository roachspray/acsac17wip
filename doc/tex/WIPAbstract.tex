\documentclass[11pt]{acmart}
\usepackage{geometry}                % See geometry.pdf to learn the layout options. There are lots.
\geometry{letterpaper}                   % ... or a4paper or a5paper or ... 
%\geometry{landscape}                % Activate for for rotated page geometry
%\usepackage[parfill]{parskip}    % Activate to begin paragraphs with an empty line rather than an indent
\usepackage{graphicx}
%\usepackage{amssymb}
\usepackage{epstopdf}
\usepackage{hyperref}
\usepackage{multicol}
\usepackage{algpseudocode,algorithm2e}

\newtheorem{theorem}{Theorem}
\newtheorem{conj}{Conjecture}
\newtheorem*{req}{Requirement}


\DeclareGraphicsRule{.tif}{png}{.png}{`convert #1 `dirname #1`/`basename #1 .tif`.png}


\title{\vspace{-4.0cm}Pruneskin: Fuzzing Hybrid Program Slices}
\author{Andrew R. Reiter\\
Veracode Inc.\\
areiter@veracode.com}
\date{}                                           

\begin{document}
\maketitle
\begin{multicols}{2}

\section{Abstract}
In recent years, there has been increased research into improving fuzz testing techniques, in particular in
developing in-process, evolutionary fuzzers \cite{DeMott} \cite{AFL} \cite{LibFuzzer}. While
initial work had the goal of increased code coverage through a feedback loop between execution and the mutation
of inputs, newer works are increasingly taking into account data flow analysis (DFA) to improve guiding the fuzzer
toward execution paths of greater interest \cite{Vuzzer}, \cite{Choronzon}. The next wave of research has been multifold
where there is an greater use of static analysis to help build input corpora \cite{StaticPA},  
fault localization techniques \cite{StaticExp}, and hybrid static and dynamic methods to support targeted (or focused)
guided fuzzing \cite{HybridVerimag} \cite{NeedleHeap}.

The goal of this work is to improve fuzzing of large, ever-changing code bases and fuzzing modules within
a code base by targeting a subprogram of a program. A subprogram here can be considered any subgraph $S$ of a flow graph of $P$.
The hope is to:
\begin{itemize}
\item Decrease the time it takes to begin fuzzing a target $S$ relative to fuzzing $P$ with the same inputs
\item Increase the time spent fuzzing a target a target $S$ relative to fuzzing $P$ with the same inputs
\item Decrease the memory footprint of the executable being fuzzed
\end{itemize}

The general approach taken is to perform hybrid slicing  \cite{HybridSlice}
in order to generate modified executables that may be fuzzed by tools such as AFL, LibFuzzer, and others. Fuzzing slices
requires that there be an additional requirement for slice generation; this is in addition to the two requirements specified
in \cite{ProgSlice} related to (1) creation of $S$ by statement deletion and (2) inputs $I$ that halt program $P$ should also halt the subprogram
$S$. The requirement is:
\begin{req}
For any input $I'$ that causes a fault in the execution of S, there exists a map $\xi : I' \mapsto I$ such that $I$ causes a fault in execution of $P$.
\end{req}

Pruneskin is the set of tools developed by the author that performs basic block-level trace point injection for dynamic control flow analysis, static value flow analysis (built on \cite{SVF}), program slice generation, and attempts at mapping of fault-causing inputs of $S$ to those for $P$. The trace point injection, value flow analysis, and program slice generation all work using LLVM's Intermediate Representation (IR) \cite{LLVM}
\cite{LLVMIR}. The high-level workflow is as follows
\begin{algorithm}[H]
\KwData{Program $P$ in IR form.\\
Target subgraph $S$ of $P$\\
Input $I$ st execution reaches S\\
Target input argument $arg$}
\hrulefill\\
$inputFlow\leftarrow valueFlow(P, arg)$\\
$P_\tau \leftarrow injectTracePoints(P)$\\
$trace \leftarrow exec(P_\tau, I)$\\
$P_S \leftarrow hybrid(P, inputFlow, trace)$\\
$crash_{S} \leftarrow fuzz(P_{S}, I)$\\
$crash_{P} \leftarrow \xi(P, P_{S}, crash_{S})$
\end{algorithm}

The current state of Pruneskin is that the tools are in place to perform dynamic control flow analysis and static value flow analysis and
use the data generated to perform hybrid program slicing; the resultant program is capable of being fuzzed as intended. Fuzz 
reachability of target time and time spent fuzzing on target have shown to be \textbf{OK, I NEED MY STATS, BUT GAH} The current focus
of work is primarily related to effective generation of maps $\xi$.

Further areas of investigation related to independent fuzzing of intervals of programs is of interest
\bibliographystyle{plain}
\bibliography{pskin}
\end{multicols}
\end{document} 
